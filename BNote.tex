%\documentclass[10pt,a4paper,slovene,openany]{book}
\documentclass[oneside,a4paper,openany,12pt]{scrbook}
%\usepackage[a4,center,cam]{crop}
%\usepackage[slovene]{babel}
%\usepackage[T1]{fontenc}
%\usepackage[pdftex]{graphicx,thumbpdf}
%\usepackage{epstopdf}
%\usepackage[math]{kurier}

\newcommand {\e}[1]{\mathrm{~#1}}
\usepackage{paralist}

\usepackage{titling}
\usepackage{amsmath,amssymb,amsfonts,nicefrac}
\usepackage{graphicx}
\usepackage{color}
\usepackage{float}
\usepackage{mathtools}
\allowdisplaybreaks
\usepackage[pdftex,colorlinks=true,citecolor=blue,linkcolor=black,urlcolor=blue,bookmarks=true]{hyperref}
\usepackage{pifont}
\usepackage{dictsym}
\usepackage{braket}
\usepackage{slashed}
\DeclareMathOperator{\arcsinh}{arcsinh}
\include{definicije}
\usepackage{enumerate}

%\usepackage[utf8]{inputenc}
%\usepackage[titles]{tocloft}
%\usepackage{setspace}

\usepackage{xr}
\externaldocument{chapterI}

%
%
\title{\huge {Measurement of the decay $B \to KK\ell\nu$}}
\author{Matic Lubej}
%
%
\begin{document}
%\maketitle
\date{Ljubljana, 2018}
\begin{titlingpage} %This starts the title page
\begin{center}
%\includegraphics[scale=1]{slike/fmf.pdf}

%\begin{large}
%\vspace{2.5cm}
\textbf{\thetitle}\\
%\vspace{0.5 cm} 
\theauthor \\
%\end{large}
\vspace{0cm}
\thedate
\end{center}
\end{titlingpage}

\pagestyle{plain}
\pagenumbering{roman}

\chapter*{Changelog}

\tableofcontents
\addtocontents{toc}{~\hfill\textbf{Page}\par}

\pagenumbering{arabic}
\chapter{Introduction}

test
\chapter{Data and MC}

test
%\chapter{Event reconstruction}

test

\chapter{Event reconstruction}

In this chapter the procedure for event reconstruction of the decay Decay is shown, starting from final state particle selection and combining particles up the chain all the way to the $B$ meson. 

\section{Final state particles selection}

Since the neutrino escapes detection, we can only reconstruct the charged tracks of our decay, which are the two kaons ($K$) and the light lepton, which is the electron ($e$) or muon ($\mu$). These are some of the particles which are commonly referred to as final state particles (FSP). Final state particles have a long lifetime and are usually the particles that we detect when they interact with the material in the detector.

It is important to limit our selection of FSP particles in order to cut down the number of particle combinations and to reduce computation time and file sizes.

\subsubsection{Leptons}

The following plots show the impact parameters $d_0$ and $z_0$, the momentum in  $\Upsilon(4S)$ center-of-mass system (CMS), and the $K/\pi$ PID information for true and fake electrons and muons, where an extra category for true electrons/muons from the signal decay is shown.

\begin{center}
PLOT
\end{center}

Based on the first plots, we can define a set of cuts:
\begin{itemize}
\item $\vert d_0 \vert < 0.1\e{cm}$,
\item $\vert z_0 \vert < 1.5\e{cm}$,
\item $p_{LAB} > 0.6\e{GeV}/c$ and $p_{CMS} \in [0.4,\,2.6]~\e{GeV}/c$ for electrons,
\item $p_{CMS} \in [0.6,\,2.6]~\e{GeV}/c$ for muons,
\end{itemize}

where the $p_{LAB}$ momentum cut for the electron case is chosen to discard a region with a sharp jump, which is assumed to come from sources like hard coded values in Belle software.

With this selection we can now determine the optimal PID cuts for electrons and muons, where we optimize with the standard definition of \textit{figure of merit} (FOM)
\begin{equation}
\label{eq:fom}
FOM = \frac{S}{\sqrt{S+B}},
\end{equation} 
where $S$ represents number of signal and $B$ the number of background candidates.

\begin{center}
PLOT
\end{center}

We now define the PID cuts for leptons:
\begin{itemize}
\item $eID > 0.9$ for electrons,
\item $muID > 0.9$ for muons.
\end{itemize}

\subsubsection{Kaons}

For the case of kaons, the cuts are very similar

\begin{center}
PLOT
\end{center}

In the same manner we can define the cuts for kaons:
\begin{itemize}
\item $\vert d_0 \vert < 0.15\e{cm}$,
\item $\vert z_0 \vert < 1.5\e{cm}$,
\item $p_{CMS} \in [0,\,2.5]~\e{GeV}/c$.
\end{itemize}

The PID optimization in this case is taken in two steps. First we optimize the $K / \pi$ PID cut, and after that the $K/p$ PID cut.

\begin{center}
PLOT
\end{center}

The resulting PID cuts are then:
\begin{itemize}
\item $K/\pi > 0.7$,
\item $K/p > 0.3$.
\end{itemize}

\section{Combination of FSP particles}

With the reconstructed kaons and leptons we can now make appropriate combinations. Due to our inability to reconstruct the neutrino, we are only able, at this point, to reconstruct the $B$ mesons in the following two channels
\begin{align*}
B^+ &\to K^+ K^- e^+, \\
B^+ &\to K^+ K^- \mu^+.
\end{align*}

In order to further guarantee the proper selection of FSP particles in a arbitrary combination, we perform a vertex fit of the three tracks. $B$ mesons have a relatively long lifetime, so they travel and decay somewhere along the z-axis of the detector, due to the detector boost, so we perform the vertex fit with an \textit{iptube} constraint, which constrains the vertex to an elongated ellipsoid along beam direction. We demand that the fit converged apply a cut on the fit probability, which was optimized with the $FOM$ function.

The resulting PID cuts are then:
\begin{itemize}
\item $chiProb > 6\times 10^{-3}$.
\end{itemize}

With the neutrino being the only missing particle on the reconstructed side and with some assumptions, we can determine the angle between the direction of the reconstructed $B$ (denoted as $Y \to K K \ell$) and the true $B$, as
\begin{align}
\mathrm{p}_\nu &= \mathrm{p}_B - \mathrm{p}_{Y}, \\
\label{eq:massnu}
\mathrm{p}_\nu^2 = m_\nu^2 &= m_B^2 + m_Y^2 - 2E_BE_Y + 2\vec{p}_B \cdot \vec{p}_Y \approx 0, \\ 
\label{eq:cosby}
\cos \left(\theta_{BY}\right) &= \frac{2E_BE_Y - m_B^2 - m_Y^2}{2\vert \vec{p}_B \vert \vert \vec{p}_Y\vert},
\end{align} 

where all the energy and momenta above are calculated in the CMS frame. The mass of the neutrino equals 0 to a very good precision, so we used it in Eq. (\ref{eq:massnu}). In addition, we can substitute the unknown energy and momentum magnitude of the $B$ meson in Eq. (\ref{eq:cosby}), $E_B$ and $\vert \vec{p}_B \vert$, with quantities from the well known initial conditions
\begin{align}
E_B &= E_{CMS} / 2,\\
\vert \vec{p}_B \vert = p_B &= \sqrt{E_B^2 - m_B^2},
\end{align} 

where $E_{CMS}$ is the total energy of the $e^+e^-$ collision in the CMS frame. For the correctly reconstructed candidates, this variable  lies in the $[-1,1]$ region, while for the background candidates the values populate also the non-physical regions. Due to detector resolution effects, we impose the cut to somewhat larger values
\begin{itemize}
\item $\vert \cos \left(\theta_{BY}\right) \vert < 1.2$.
\end{itemize}

\begin{center}
PLOT
\end{center}

\section{Loose neutrino reconstruction}

As was already mentioned, the neutrinos in the event escape the detector, so we cannot determine it's four-momentum. However, due to the detectors geometry, which almost completely covers the full solid angle, and due to well known initial conditions of the $\Upsilon(4S)$ meson, it is possible to determine the kinematics of the neutrino by summing up all the FSP particles four-momenta in the event. This is known as the \textit{untagged} method.

The total missing four-momentum in the event can be determined as
\begin{align}
\mathrm{p}_{miss} &= \mathrm{p}_{\Upsilon(4S)} - \sum_i^{\mathrm{Event}}\left(E_i,\,\vec{p}_i \right),\\
\mathrm{p}_{miss} &= \mathrm{p}_{\Upsilon(4S)} - \left(\mathrm{p}_{Y} -\sum_i^{\mathrm{Rest~of~event}}\left(E_i,\,\vec{p}_i \right)\right).
\end{align}

where the summation runs over all charged and neutral particles in the event. We can define a new variable, $m_{miss}^2$, which is the square of the missing mass. If signal side neutrino is the only missing particle in the event, then this variable should be equal to zero
\begin{align}
\label{eq:nuold}
\mathrm{p}_\nu &= \mathrm{p}_{miss} = \left(E_{miss},\,\vec{p}_{miss} \right),\\
m_{miss}^2 &= \mathrm{p}_{miss}^2 = \mathrm{p}_{\nu}^2 = m_\nu^2 \approx 0.
\end{align}

Since the detector is not perfect, the distribution of the $m_{miss}^2$ variable has a non-zero width. Additionally, tails are introduced as soon as we have multiple missing neutrinos, other neutral undetected particles such as $K_L^0$, or simply missing tracks due to detection failure.

\begin{center}
PLOT
\end{center}

The main uncertainty in neutrino four-momentum, defined as in Eq. (\ref{eq:nuold}) comes from energy uncertainty. It is a common practice to substitute the missing energy with the magnitude of the missing momentum, since the momentum resolution is much better, thus redefining the neutrino four-momentum to
\begin{equation}
\label{eq:nunew}
\mathrm{p}_\nu = \left(\vert \vec{p}_{miss} \vert,\,\vec{p}_{miss} \right).
\end{equation}

With our newly defined neutrino four-momentum, we can add it to the four-momentum of the $Y(KK\ell)$ candidate to obtain the full $B$ meson four-momentum and calculate the traditional $M_{BC}$ and $\Delta E$ variables
\begin{align}
\Delta E &= E_B - E_{CMS}/2,\\
M_{BC} &= \sqrt{\left(E_{CMS}/2\right)^2 - p_B^2}.
\end{align}

Since the final fit will be performed on these two variables, we define two regions that will be used
\begin{itemize}
\item Fit region: $M_{BC} \in [5.1,\,5.3]\e{GeV}/c^2$ and $\Delta E \in [-1,\,1]\e{GeV}$,
\item Signal enhanced region: $M_{BC} \in [5.X,\,5.3]\e{GeV}/c^2$ and $\Delta E \in [-X,\,X]\e{GeV}$.
\end{itemize}

\begin{center}
PLOT
\end{center}


\section{Rest of event clean-up}

\section{Selection after clean-up}

\end{document}


