% Red-black tree
% Author: Madit
\documentclass{article}
\usepackage{tikz}
%%%<
\usepackage{verbatim}
\usepackage[active,tightpage]{preview}
\PreviewEnvironment{tikzpicture}
\setlength{\PreviewBorder}{10pt}%
%%%>
\begin{comment}
:Title: Red-black tree
:Tags: Trees;Graphs
:Author: Madit
:Slug: red-black-tree

A red-black tree is a special type of binary tree, used in computer science
to organize pieces of comparable data, such as text fragments or numbers.
(Wikipedia)
\end{comment}
\usetikzlibrary{arrows}

\tikzset{
  treenode/.style = {align=center, inner sep=0pt, text centered},
  arn_n/.style = {treenode, circle, white, draw=black,
    fill=black, text width=1.5em},% arbre rouge noir, noeud noir
  arn_r/.style = {treenode, circle, white, draw=black, fill=black,
    text width=1.5em, very thick},% arbre rouge noir, noeud rouge
  arn_x/.style = {treenode, rectangle, draw=black,
    minimum width=0.5em, minimum height=0.5em}% arbre rouge noir, nil
}

\begin{document}
\begin{tikzpicture}[scale=0.5,-latex,level/.style={sibling distance = 3cm/#1,
  level distance = 2.5cm}] 
\node [arn_n] at (8,0) {1} 
    child{ node [arn_r] {2} 
            child{ node [arn_n] {3} 
							child{}
							child{}
            }
            child{ node [arn_n] {3}
							child{ }
							child{ }
            }                            
    }
    child{ node [arn_r] {2}
            child{ node [arn_n] {3} 
							child{}
							child{}
            }
            child{ node [arn_n] {3}
							child{}
							child{}
            }
		}
; 

\node at (0,-9) {
\begin{tabular}{c}
$\vdots$\\
$\texttt{nLevels}$
\end{tabular}
};

\node [arn_n] {1}
    child{ node [arn_r] {2} 
            child{ node [arn_n] {3} 
							child{}
							child{}
            }
            child{ node [arn_n] {3}
							child{ }
							child{ }
            }                            
    }
    child{ node [arn_r] {2}
            child{ node [arn_n] {3} 
							child{}
							child{}
            }
            child{ node [arn_n] {3}
							child{}
							child{}
            }
		}
; 

\node at (8,-9) {
\begin{tabular}{c}
$\vdots$\\
$\texttt{nLevels}$
\end{tabular}
};

\node at (13,-4) {$\cdots$ \texttt{nTrees}};

\end{tikzpicture}
\end{document}
