\documentclass[headings=standardclasses,headings=big,oneside,a4paper,openany,12pt]{scrbook}

\newcommand {\e}[1]{\mathrm{~#1}}
\newcommand {\E}[1]{\times 10^{#1}}
\newcommand {\vars}{$\Delta E$ and $M_{BC}$}
\newcommand {\btbii}{\texttt{B2BII}}
\newcommand {\decaya}{$B \to K K \ell \nu$}
\newcommand {\decayb}{$B^+ \to K^+ K^- \ell^+ \nu$}

%\usepackage{biblatex}
%\bibliography{mybib.bib} 
\usepackage[english]{babel}% Recommended
\usepackage{csquotes}% Recommended
\usepackage[sorting=none,firstinits=true,backend=bibtex]{biblatex}
\addbibresource{mybib.bib}% Syntax for version >= 1.2

\usepackage{paralist}
\usepackage{caption}
\usepackage{cancel}

\usepackage{longtable}

\setlength{\parskip}{1em}%
\setlength{\parindent}{0cm}

\usepackage{titling}
\usepackage{amsmath,amssymb,amsfonts,nicefrac}
\usepackage{graphicx}
\usepackage{color}
\usepackage{float}
\usepackage{mathtools}
\allowdisplaybreaks
\usepackage[pdftex,colorlinks=true,citecolor=blue,linkcolor=black,urlcolor=blue,bookmarks=true]{hyperref}
\usepackage{dictsym}
\usepackage{braket}
\usepackage{slashed}
\DeclareMathOperator{\arcsinh}{arcsinh}
\usepackage{enumerate}
\usepackage{array}
\setlength{\extrarowheight}{.5ex}

\usepackage{lineno}
\linenumbers

\usepackage{subfigure}

\begin{document}


\chapter{Experimental Setup}
The data used in this analysis were produced in $e^+e^-$ collisions at the KEKB accelerator and collected with the Belle detector. The experiment was hosted at the High Energy Accelerator Research Organization (KEK) in Tsukuba, Japan. The experiment ran
from years 1999 to 2010, collecting data at and near the energy of the $\Upsilon(4S)$ resonance. This chapter briefly describes the accelerator and the detector. The descriptions are based on detailed reports from [X] and [X].


\section{KEKB Accelerator}
KEKB is an asymmetric $e^+e^-$ collider, composed roughly of an electron source and a positron target, a linear accelerator (Linac) and two separate main rings with a circumference of about $3\e{km}$ as shown in Figure X. Electrons are first produced by a thermal electron gun and accelerated in the Linac to an energy of about $8\e{GeV}$. Part of the electrons collide with a tungsten target to produce positrons, which are accelerated in the Linac to an energy of about $3.5\e{GeV}$. Electron and positron beams are injected into the high- (HER) and low energy ring (LER) where they collide at a single interaction point (IP) at an angle of about $22\e{mrad}$. The combined centre-of-mass (CM) energy of the collision corresponds to the mass of the $\Upsilon(4S)$ resonance
\begin{equation}
E_{CM} = 2\sqrt{E_{e^+}E_{e^-}} = m_{\Upsilon(4S)}c^2 \approx 10.58\e{GeV}.
\end{equation}

PLOT

%TODO describe what is Bhabha?
The $\Upsilon(4S)$ state is produced only in a fraction of all collisions, but when it is produced, it predominantly decays to a pair of charged or neutral $B$ mesons. This setup was chosen in accordance with the main goal of the experiment, which was to study CP
violation in the $B$ meson system. In other cases the collisions of $e^+e^-$ result in Bhabha scattering, two-photon events, muon or tau lepton pair production, and quark pair production of $q \bar q$, where $q=u,\,d,\,s$ or $c$. Table X shows the cross-sections for all mentioned interactions in collisions of $e^+e^-$.
In addition to the nominal CM energy, the experiment collected data also at energies
corresponding to other $\Upsilon(nS)$ resonances, where $n = 1,\,2,\,3,\,5$, and also at energies below the resonances.

% TODO: check tables style and caption, cite belle detector X


\begin{center}
	\begin{tabular}{c|c}
		Interaction & Cross-section $[\mathrm{nb}]$ \\ 
		\hline
		$\Upsilon(4S) \to B \bar B$ & $1.2$ \\
		$q \bar q,~q \in [u,d,s,c]$ & $2.8$ \\
		$\mu^+\mu^-,~\tau^+\tau^-$ & $1.6$ \\
		Bhabha scattering (within detector acceptance)& $44$ \\
		Other QED processes (within detector acceptance)& $\sim 17$ \\
		\hline
		Total & $\sim 67$ 
	\end{tabular} 
\end{center}

KEKB achieved the world-record for the peak luminosity of $2.11\E{34}\e{cm^{-2}s^{-1}}$, twice as much as the designed prediction, and the total integrated luminosity of $1041\e{fb^{-1}}$. Of the full Belle dataset, about $711\e{fb^{-1}}$ of data were taken at the $\Upsilon(4S)$ energy of $10.58\e{GeV}$, which corresponds to about $771\E{6}$ $B \bar B$ meson pairs.


\section{Belle Detector}
The Belle detector is a magnetic mass spectrometer which covers a large solid angle. It is designed to detect remnants of $e^+e^-$ collisions. The detector is configured around a $1.5\e{T}$ superconducting solenoid and iron structure surrounding the interaction point (IP). The 4-momentum of the decaying $B$ mesons and it's decayed daughter particles are determined via a series of sub-detector systems, which are installed in an onion-like shape. Short-lived particle vertices are measured by a silicon vertex detector (SVD) situated outside of a cylindrical beryllium beam pipe. Long-lived charged particle momentum is measured via tracking, which is performed by a wire drift chamber (CDC). Particle identification is provided by energy-loss measurements in CDC, aerogel Cherenkov counters (ACC) and time-of-flight counters (TOF) situated radially outside of CDC. Particles producing electromagnetic showers deposit energy in an array of CsI(Tl) crystals, known as the electromagnetic calorimeter (ECL), which is located inside the solenoid coil. Muons and $K_L$ mesons (KLM) are identified by arrays of resistive plate counters in the iron yoke. 

The coordinate system of the Belle detector originates at the IP, with the $z$ axis pointing in the opposite direction of the positron beam, the $x$ axis pointing horizontally out of the ring, and the $y$ axis perpendicular to the aforementioned ones. Th electron beam crosses the positron beam at an angle of about $22^\circ$. The polar angle $\theta$ covers the region between $17^\circ \leq \theta \leq 150^\circ$, while the cylindrical angle $\varphi$ covers the full range $0^\circ \leq \varphi \leq 360^\circ$, amounting to about $92~\%$ coverage of the full solid angle.


%In our analysis, key features of the Belle detector are as follows.
%•
%Good particle detection efficiency for 4-
%π
%region to collect all the particles in an
%e
%+
%e
%−
%collision.  This is important not only for the good signal reconstruction efficiency,
%but also for background reduction since un-detected particles from various
%B
%decay
%modes are the main source of the background.
%•
%Good reconstruction efficiency for low-momentum particles.  Since decay particles
%from
%D
%∗
%mesons and
%τ
%leptons are produced in a chain of multi-body decays, their
%momenta are about 0.7 GeV
%/c
%on average and lower than 1.5 GeV
%/c
%.
%•
%Good  momentum  resolution  in  the  tracking  devices  and  energy  resolution  in  the
%ECL.
%•
%Good PID performance for charged particles.

%\subsection{Extreme forward calorimeter (EFC)}

\subsection{Silicon Vertex Detector}
SVD is the inner-most part of the Belle detector and serves the purpose of measuring the decay vertices of decaying particles. The precision of the subsystem is about $100\e{\mu m}$, which is important for measuring the difference in $z$-vertex positions of the $B$ mesons in time-dependent CP violation studies. The main part of the SVD are the double-sided silicon detectors (DSSD).

During the data taking period, two configurations were od the SVD have been used. The first, SVD1, has three layers of DSSD detectors, positioned at $30$, $45.5$ and $60\e{mm}$ away from the IP. They compose a ladder-like structure, covering the polar angle of $23^\circ < \theta < 140^\circ$. This configuration was used from the beginning od the experiment until 2003, when a dataset of about $1.52\E{8}$ pairs of $B \bar B$ mesons were recorded. After that a new configuration was used, SVD2, which was operational until the end of data taking, measuring about $6.20\E{8}$ pairs of $B \bar B$ mesons. The SVD2 has 4 layers of DSSD detectors positioned at $20$, $43.5$, $70$ and $80\e{mm}$ away from the IP and covered the polar angle of $17^\circ < \theta < 150^\circ$. The first layer was moved closer to the IP, which greatly improved the sub-system precision, due to multiple-Coulomb scattering affecting resolution more as the distance from the IP increases.

The momentum and angular dependence of the impact parameter resolution are shown in Figure X for both SVD configurations and are well represented by the expressions $p\beta \sin^{5/2}\theta$ and $p\beta \sin^{3/2}\theta$ for the direction parallel and perpendicular to the $z$ axis, respectively, where $p$ is the particle momentum, $\theta$ is the polar angle, and $\beta=v/c$.

PLOT

%reference to [25]
%A. Abashian et al. The Belle Detector.
%Nucl. Instrum. Method Phys. Res., Sect.
%A
%, 479:117–232, 2002. doi: 10.1016/S0168-9002(01)02013-7

\subsection{Central Drift Chamber}

CDC is a large-volume tracking device located at the central part of the Belle detector. It has a cylindrical structure with a radius of $88\e{cm}$, length of $2.4\e{m}$ and acceptance equal to the one of SVD2. The chamber has a total of $8400$ wires, which are positioned in $50$ layers and describe nearly square configuration. There are two types of wires -- field wires for producing the electrical field, and sense wires for detecting the particles. Odd-numbered wire layers are oriented in the $z$ direction and provide measurement of the transverse momentum $p_t$, while even-numbered wires are inclined with respect to the $z$ axis by a small angle of $\pm 50\e{mrad}$ to allow for measuring of the polar angle of the track. The resolution of the transverse momentum is
$$\sigma(p_T) = 0.201\%p_t \oplus 0.290\%\beta.$$
The space between the wires is filled with a gas mixture of $1:1$ helium-ethane, a low-$Z$ gas in order to minimize multiple-Coulomb scattering contributions to momentum resolution. It also has a small cross section of the photoelectric effect, which is important to reduce background electrons induced by the synchrotron radiation from the beam.
%TODO check equations

Charged particles which pass the CDC wire frame cause gas ionization. The produced electrons drift toward the sense wires with great acceleration due to the strong electric field close to the wire. The accelerated electrons collide with the gas and produce secondary ionizations and so on, which results in an electron avalanche, a process which increases the signal by more many orders of magnitude. The primary electrons also have a specific drift velocity, which allows us to relate the measured pulse height and drift time to the energy deposit of the particle as well as the distance from the sense wire. This information is important for calculating the energy loss $\mathrm{d}E/\mathrm{d}x$. $\mathrm{d}E/\mathrm{d}x$ as a function of momentum is different for different particles as shown in Figure X. This allows for identification purposes of particles, specifically kaons and pions. In the momentum region less than $0.8\e{GeV}/c$ $\mathrm{d}E/\mathrm{d}x$ enables a separation between kaons and pions up to $3\sigma$.

PLOT

\subsection{Time-of-Flight Counter}

The purpose of the TOF subdetector is particle identification in the momentum region $0.8\e{GeV}/c < p < 1.2\e{GeV}/c$, especially for kaons and pions. It measures the time interval between the $e^+e^-$ collision and the passage of the particle through TOF with a resolution of about $100\e{ps}$. The mass of a particle can be inferred via the relation
\begin{equation}
m^2 = \left( \frac{1}{\beta^2}-1\right)p^2 = \left( \frac{T^2c^2}{L^2}-1\right)p^2,
\end{equation}
%TODO: number equations
where $T$ is the measured time interval, $L$ is the charged particle trajectory length from the IP and $p$ is the charged particle momentum, determined by SVD and CDC. Figure X shows the mass distribution for charged tracks measured by TOF in hadron events. Clear peaks corresponding to pions, kaons and protons can be seen.

There are 64 TOF modules in the barrel region, covering the polar angle of $33^\circ < \theta < 121^\circ$. One TOF module consists of two long plastic scintillator bars, 4 fine-mesh photo-multiplier tubes (PMT) at the 4 ends of the bars, and a trigger scintillation counter (TSC), where the latter provides additional trigger information.

PLOT

%TODO: separation?

\subsection{Aerogel Cherenkov Counter} %todo: all caps titles
TOF is not capable of performing good PID above $1\e{GeV}/c$ momentum, since $\beta$ is almost equal to 1. For higher momentum in the region $1.0\e{GeV}/c < 4.0\e{GeV}/c$, the ACC is introduced. It is a threshold-type Cherenkov counter which utilizes the fact that particles emit Cherenkov light if the particle speed is greater than the speed of light in the passing medium. The threshold velocity $\beta$ of a given particle for Cherenkov radiation is
\begin{equation}
\beta \leq \frac{1}{n},
\end{equation}
where $n$ is the refractive index of the medium. The refractive indices in the ACC are such that, due to different masses, pions will emit Cherenkov light and kaons will not, due to different masses of the particles. 

The ACC is introduced in the barrel region with 960 separate module, covering a polar angle of $34^\circ < \theta < 127^\circ$ and 228 modules in the forward endcap regions, with the polar angle coverage of $17^\circ < \theta < 34^\circ$. Each module consists of an aluminum encased block of silica aerogel and one or two fine-mesh PMTs encased on each block to detect Cherenkov light pulses. Due to the polar angle dependence of the particle momentum, 6 different refractive indices are chosen for the aerogel material, ranging from $1.010$ up to $1.030$. The layour of the ACC is shown in Figure X.

%TODO: efficiency?

\subsection{Electromagnetic Calorimeter}
The ECL provides measurement of position and energy deposit of particles, especially electrons and photons, where the latter are not measured by any of the subsystems described so far. It also provides complimentary particle identifications for electrons versus pions.

%TODO: check spelling
This subdetector consists of a highly segmented array of thallium-doped cesium iodide (CsI(Tl)) tower-shaped crystals, each pointing towards the IP. Each crystal is about $30\e{cm}$ long with a width from $44.5\e{mm}$ to $65\e{mm}$ in the barrel, and from $44.5\e{mm}$ to $82\e{mm}$ in the endcaps. Out of a total of 8736 crystals, 6624 are positioned in the barrel region and 1152 (960) in the forward (backward) endcaps. The inner radius of the barrel section is about $1.25\e{m}$, while the endcaps are positioned at $-1.0\e{m}$ and $2.0\e{m}$ from the IP in the $z$ direction. The polar angle coverage of the barrel region is $32.2^\circ < \theta < 128.7^\circ$ and for the encaps $12.4^\circ < \theta < 31.4^\circ$ and $130.7^\circ < \theta < 155.1^\circ$.

When an electron or a photon hits a crystal, it produces an electromagnetic shower, a result of the bremsstrahlung and pair-production effects. Heavier charged particles do not interact in the same way and deposit only a small amount of energy by ionization effects. Electron identification can be performed by looking at the ratio $E/p$, which is close to unity for electrons, while lower for heavier charged particles. The average energy resolution is $1.7\%$ and is given by
\begin{equation}
\frac{\sigma_E}{E} = \frac{0.0066\%}{\left(E/1\e{GeV}\right)}\oplus\frac{1.53\%}{\left( E/1\e{GeV}\right)^{1/4}}\oplus 1.18\%,
\end{equation}
while the resolution of the position measurement is
\begin{equation}
\sigma_{pos} = 0.27\e{mm}+\frac{3.4\e{mm}}{\left( E/1\e{GeV}\right)^{1/2}} + \frac{1.8\e{mm}}{\left( E/1\e{GeV}\right)^{1/4}}
\end{equation}

%TODO: add plots, efficiencies...

\subsection{$K_L^0/\mu$ Detector}
The KLM detector is used for detection of high-penetration particles such as $K_L^0$ and $\mu$ for momenta larger than $0.6\e{GeV}/c$. The setup covers the polar angle of $20^\circ < \theta 155^\circ$. Detection of $K_L^0$ particles is troublesome, since they are neutral and since they have a small material interaction probability, therefore a lot of material is needed in the KLM. To provide detection of both kinds of particles, hadronic and neutral, as well as electromagnetically and hadronically interacting, the KLM is constructed as a sampling calorimeter, which consists of 15 layers of $3.7\e{cm}$ thick resistive-plate counters (RPC) with 14 layers of $4.7\e{cm}$ thick iron plates between them. A single RPC module consists of two parallel plate electrodes, two glass panels, and gas in between. A charged particle passing the gas gap initiates a local discharge of the plates, which in turn induces signal to record the time and location of ionization. Hadrons interacting with the iron plates may produce a shower of ionizing particles, which are then detected by the RPCs. 

The $K_L^0$ particle can be distinguished from other charged hadrons because they have no matched track in the CDC. The flight direction can also be inferred from the hit locations in the consecutive RPCs. On the other hand, muons do have matched tracks in the CDC, but they do not interact strongly and do not produce hadronic showers in the KLM and can be recognized in this way.

efficiency etc, plots?


\subsection{Trigger System and Data Acquisition}
\subsection{Particle Identification}

\printbibliography

\end{document}


