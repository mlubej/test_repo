\chapter{Rest of event clean-up}
\label{ch:roe}

Continuing from Section \ref{sec:loose-neutrino-reconstruction}, the description of the ROE clean-up process is described here. 

Training the MVA classifiers follows the same recipe for all the steps in this chapter. For each step we run $B$ meson reconstruction on Signal MC with a generic companion $B$ meson. This way the produced weight files are less likely to be signal-side dependent and can be used also for untagged analyses of other decays. For every correctly reconstructed signal $B$ meson we save the necessary information for each MVA step (i.e. properties of ROE clusters). Only correctly reconstructed $B$ candidates are chosen here, to prevent leaks of information from the signal side to the ROE side.

\section{Clusters clean-up}

Photons originate from the IP region, travel to the ECL part of the detector in a straight line and produce a cluster. The direction of the photon is determined via the location of the cluster hit in the ECL and the energy of the photon is directly measured via the deposited energy. This way the four-momentum of photons is determined and used in Eq. (\ref{eq:ROEloop}).

Most of the photons in events with $B$ mesons come from $\pi^0 \to \gamma \gamma$ decays. However, a lot of hits in the ECL are also created by photons coming from beam-induced background or secondary interactions with the detector material. Such photons add extra energy and momentum which spoils our measured quantities.

In the first step of the clusters clean-up we train an MVA which recognizes good $\pi^0$ candidates. The output of this classifier is then applied to photons and represents a sort of a $\pi^0$ origin probability, which is used as an additional classifier variable in the next step of the clean-up.

\subsection{$\pi^0$ MVA training}

The dataset of $\pi^0$ candidates contains
\begin{itemize}
\item 387125 target candidates,
\item 416019 background candidates,
\end{itemize}
where the definition of target is that both photon daughters that were used in the reconstruction of the $\pi^0$ are actual photons and real daughters of the $\pi^0$ particle. We use $\pi^0$ candidates from the converted Belle particle list and select those with invariant mass in the range of $M \in [0.10,~0.16]\e{GeV}$. After that we perform a mass-constrained fit on all candidates, keeping only the ones for which the fit converged. 

The input variables used in this MVA are
\begin{itemize}
\item $p$ and $p_{CMS}$ of $\pi^0$ and $\gamma$ daughters,
\item fit prob. of the mass-constrained fit, invariant mass and significance of mass before and after the fit,
\item angle between the photon daughters in the CMS frame,
\item cluster quantities for each daughter photon
	\begin{itemize}
	\item $E_9/E_{25}$,
	\item theta angle,
	\item number of hit cells in the ECL,
	\item highest energy in cell,
	\item energy error,
	\item distance to closest track at ECL radius.
	\end{itemize}
\end{itemize}

The classifier output variable is shown in Figure \ref{fig:ROE_pi0}.

\begin{figure}[H]
\centering
\captionsetup{width=0.8\linewidth}
\includegraphics[width=\linewidth]{fig/ROECleanup_pi0}
\caption{Classifier output of the $\pi^0$ training for signal and background $\pi^0$ candidates.}
\label{fig:ROE_pi0}
\end{figure}

The distributions for all input variables and their correlations for signal and background candidates can be found in Appendix A for all steps of the ROE clean-up.


\subsection{$\gamma$ MVA training}

In this MVA training we take the $\pi^0$ classifier output of the previous training as an input in order to train a classifier to distinguish between good and bad photons. The $\pi^0$ probability information from the previous step is applied to all photon pairs which pass the same $\pi^0$ cuts as defined in the previous step. Since it's possible to have overlapping pairs of photons, the $\pi^0$ probability is overwritten in the case of a larger value, since this points to a greater probability of a correct photon combination. On the other hand, some photon candidates fail to pass the $\pi^0$ selection, these candidates have a fixed value of $\pi^0$ probability equal to zero.

The dataset of $\gamma$ candidates contains
\begin{itemize}
\item 324781 target candidates,
\item 333353 background candidates,
\end{itemize}
where the definition of target is that the photon is an actual photon which is related to a primary MC particle. This tags all photon particles from secondary interactions as background photons. We use the converted $\gamma$ candidates from the existing Belle particle list. 

The input variables used in this MVA are
\begin{itemize}
\item $p$ and $p_{CMS}$ of $\gamma$ candidates,
\item $\pi^0$ probability,
\item cluster quantities
	\begin{itemize}
	\item $E_9/E{25}$,
	\item theta angle,
	\item number of hit cells in the ECL,
	\item highest energy in cell,
	\item energy error,
	\item distance to closest track at ECL radius.
	\end{itemize}
\end{itemize}

The classifier output variable is shown in Figure \ref{fig:ROE_gamma}.

\begin{figure}[H]
\centering
\captionsetup{width=0.8\linewidth}
\includegraphics[width=\linewidth]{fig/ROECleanup_gamma}
\caption{Classifier output of the $\gamma$ training for signal and background $\gamma$ candidates.}
\label{fig:ROE_gamma}
\end{figure}

With the final weights for photon classification in hand, we apply them to the photon particle list. The cut optimization is shown in Figure \ref{fig:ROE_gamma_opt} (left), with the optimal cut on the $\gamma$ classifier output at
\begin{itemize}
\item $BDT_\gamma > 0.519$.
\end{itemize}

Figure \ref{fig:ROE_gamma_opt} (right) shows the LAB frame momentum of the photons before and after the cut in logarithmic scale. The signal efficiency and background rejection at this clean-up cut are
\begin{itemize}
\item Signal efficiency: $\epsilon_{SIG} = 82.6~\%$,
\item Background rejection: $1-\epsilon_{BKG} = 81.7~\%$.
\end{itemize}

\begin{figure}[H]
\centering
\captionsetup{width=0.8\linewidth}
\includegraphics[width=\linewidth]{fig/ROECleanup_gamma_opt}
\caption{The $\mathrm{FOM}$ of the classifier output optimization (left) and  momentum magnitude in the LAB frame of signal and background photon candidates before and after the optimal cut (right).}
\label{fig:ROE_gamma_opt}
\end{figure}

The event is now considered to be clean of extra clusters.

\section{Tracks clean-up}

Charged particles leave hits in the detector, which are then grouped into tracks by advanced tracking algorithms. The track is fitted and the track momentum is determined. With the help of particle identification information (PID), we are able to make an intelligent decision about the mass hypothesis of the particle and thus reconstruct the charged particle's four-momentum, which is then added in the loop in Eq. (\ref{eq:ROEloop}).

Most of the quality (good) tracks, which come from physics event of interest, come from the IP region, where the collisions occur. Cleaning up the tracks is a more complex procedure than cleaning up the clusters. The following facts need to be taken into account
\begin{enumerate}[(a)]
\item good tracks can also originate away from the IP region, due to decays of long-lived particles, such as $K_S^0 \to \pi^+ \pi^-$,
\item charged particles from background sources produce extra tracks, or duplicates,
\item low momentum charged particles can curl in the magnetic field and produce multiple tracks,
\item secondary interactions with detector material or decays of particles in flight can produce "kinks" in the flight directory, resulting in multiple track fit results per track.
\end{enumerate}

Schematics of all the cases mentioned above are shown in Figure \ref{fig:track_cleanup}.

\begin{figure}[H]
\centering
\subfigure[]{
\includegraphics[width=0.49\linewidth]{texfig/V0}}%
\subfigure[]{
\includegraphics[width=0.49\linewidth]{texfig/background}}\\
\subfigure[]{
\includegraphics[width=0.49\linewidth]{texfig/curler}}%
\subfigure[]{
\includegraphics[width=0.49\linewidth]{texfig/decay}}%
\caption{(a) Tracks from long-lived neutral particles, which decay away from the IP region, (b) Random tracks from background which are reconstructed, (c) Low-momentum particles which curl in the magnetic field, (d) in-flight decays of particles, which produce a kink in the trajectory.}
\label{fig:track_cleanup}
\end{figure}

It is obvious that tracks from the same momentum source should only be taken into account once, or, in case of background tracks, not at all. Such tracks will from this point on be denoted as \textit{extra} tracks, because they add extra four-momentum to our final calculations in Eq. (\ref{eq:ROEloop}). At the same time, we have to take care that we don't identify \textit{good} tracks as \textit{extra} tracks. Both of these cases have negative impacts on the final resolution of all variables which depend on information from ROE.

\subsection{Tracks from long-lived particles}

The first step in tracks clean-up is taking care of tracks from long-lived particles. Here we only focus on $K_S^0$, since they are the most abundant. This step is necessary because the $\pi^\pm$ particles, coming from the $K_S^0$ decays, have large impact parameters, which is usually a trait of background particles. In order to minimize confusion from the MVA point-of-view, these tracks are taken into account separately.

We use the converted $K_S^0$ candidates from the existing Belle particle list and use a pre-trained Neural Network classifier result in order to select only the good $K_S^0$ candidates. Figure \ref{fig:ROE_V0} shows the distribution of the $K_S^0$ invariant mass for signal and background candidates, before and after the classifier cut. The momentum of selected $K_S^0$ candidates is added to the ROE, while the daughter tracks are discarded from our set.

\begin{figure}[H]
\centering
\captionsetup{width=0.8\linewidth}
\includegraphics[width=\linewidth]{fig/ROECleanup_V0}
\caption{Invariant mass of the $K_S^0$ candidates before (dashed lines) and after (solid lines) the cut on the Neural Network classifier for signal (green) and background candidates (red). Signal peaks at nominal $K_S^0$ mass, while background covers a wider region.}
\label{fig:ROE_V0}
\end{figure}

The signal efficiency and background rejection for $K_S^0$ candidates after this cut and on the full range are
\begin{itemize}
\item Signal efficiency: $\epsilon_{SIG} = 80.7~\%$,
\item Background rejection: $1-\epsilon_{BKG} = 99.4~\%$.
\end{itemize}

\subsection{Duplicate tracks}
All good tracks at this point should be coming from the IP region, since we took care of all the good tracks from long-lived particle decays, therefore we apply a cut on impact parameters for all the remaining tracks
\begin{itemize}
\item $\vert d_0 \vert < 10\e{cm}$ and $\vert z_0 \vert < 20\e{cm}$
\end{itemize} 

and proceed with the clean-up of track duplicates.

\subsubsection{Defining a duplicate track pair}

In this step we wish to find a handle on secondary tracks from low momentum curlers and decays in flight. The main property for these cases is that the angle between such two tracks is very close to $0^\circ$ or $180^\circ$, since tracks deviate only slightly from the initial direction, but can also be reconstructed in the opposite way. Figure \ref{fig:ROE_dupAngleInit} shows the distribution of the angle between two tracks in a single pair for random track pairs and duplicate track pairs, where the latter were reconstructed as two same-sign or opposite-sign tracks.

\begin{figure}[H]
\centering
\captionsetup{width=0.8\linewidth}
\includegraphics[width=\linewidth]{fig/ROECleanup_dup_angle_initial}
\caption{Distribution of the angle between two tracks in a single pair for random track pairs (green) and duplicate track pairs, where the latter were reconstructed as two same-sign (blue) or opposite-sign tracks (red).}
\label{fig:ROE_dupAngleInit}
\end{figure}

If the particle decayed mid-flight or produced multiple tracks due to being a low-momentum curler, then, as the name suggests, these particles most likely had low momentum in the transverse direction, $p_T$. Since both tracks originate from the same initial particle, the momentum difference should also peak at small values. Figure \ref{fig:ROE_dupPt} shows the momentum and momentum difference of tracks which belong to a random or a duplicate track pair.

\begin{figure}[H]
\centering
\captionsetup{width=0.8\linewidth}
\includegraphics[width=\linewidth]{fig/ROECleanup_dup_pt}
\caption{Distribution of transverse momentum $p_T$ (left) and transverse momentum difference $\Delta p_T$ (right) for all tracks coming from random (green) or duplicate track pairs (red). The plot on the right already includes the cut on $p_T$ from the plot on the left.}
\label{fig:ROE_dupPt}
\end{figure}

We impose a cut of
\begin{itemize}
\item $p_T < 0.5\e{GeV}/c$,
\item $\vert \Delta p_T \vert < 0.1\e{GeV}/c$,
\end{itemize}

in order to cut down the number of random track pairs, while retaining a high percentage of duplicate track pairs. After all the cuts defined in this chapter, the final distribution of the angle between two tracks is shown in Figure \ref{fig:ROE_dupAngleFinal}.

\begin{figure}[H]
\centering
\captionsetup{width=0.8\linewidth}
\includegraphics[width=\linewidth]{fig/ROECleanup_dup_angle_final}
\caption{Distribution of the angle between two tracks in a single pair after applying the selection cuts defined in this subsection. The distributions are shown for random track pairs (green) and duplicate track pairs, where the latter were reconstructed as two same-sign (blue) or opposite-sign tracks (red).}
\label{fig:ROE_dupAngleFinal}
\end{figure}

\subsubsection{Training the duplicate track pair MVA}
\label{ss:trackMVA}

This final sample of track pairs is now fed into an MVA, which is trained to recognize duplicate track pairs over random ones. The dataset contains
\begin{itemize}
\item 215601 target candidates,
\item 311124 background candidates,
\end{itemize}
where the definition of target is that the track pair is a duplicate track pair. 

The input variables used in this MVA are
\begin{itemize}
\item angle between tracks,
\item track quantities
	\begin{itemize}
	\item impact parameters $d_0$ and $z_0$,
	\item transverse momentum $p_T$,
	\item helix parameters and helix parameter errors of the track,
	\item track fit $p$-value,
	\item number of hits in the SVD and CDC detectors
	\end{itemize}
\end{itemize}

The classifier is able to distinguish between random and duplicate track pairs in a very efficient manner, as shown in Figure \ref{fig:ROE_dupBDT}.

\begin{figure}[H]
\centering
\captionsetup{width=0.8\linewidth}
\includegraphics[width=\linewidth]{fig/ROECleanup_dup}
\caption{Classifier output of the track pair training for random track pairs and duplicate track pairs.}
\label{fig:ROE_dupBDT}
\end{figure}

The $\mathrm{FOM}$ function for optimal cut selection is shown in Figure \ref{fig:ROE_dupOpt} (left), along with the angle between the two tracks before and after the optimal cut (right). The optimal cut for duplicate track selection is
\begin{itemize}
\item $BDT_{duplicate} > 0.99915.$
\end{itemize}

\begin{figure}[H]
\centering
\captionsetup{width=0.8\linewidth}
\includegraphics[width=\linewidth]{fig/ROECleanup_dup_opt}
\caption{The optimization of the $\mathrm{FOM}$ function for the cut on classifier output (left) and distribution of the angle between two tracks in a single pair before (dashed) and after (solid) applying the optimal cut on the output classifier for random and duplicate track pairs (right).}
\label{fig:ROE_dupOpt}
\end{figure}

\subsubsection{Defining duplicate tracks}

What remains now is to decide which track from the duplicate track pair to keep and which to discard. For this purpose we apply duplicate pair-level information to each track in the pair in the form of
\begin{equation}
\Delta f = f_{this} - f_{other},
\end{equation}

where $f$ is an arbitrary variable from the list of track quantities in Subsection \ref{ss:trackMVA}. From the point-of-view of \textit{this} track, a track is more $duplicate$-like if the following is true
\begin{itemize}
\item $\Delta d_0,\,\Delta z_0 > 0$ (\textit{this} track further away from the IP region),
\item $\Delta p_T,\,\Delta p_Z < 0$ (\textit{this} track has lower momentum),
\item $\Delta N_{SVD},\,\Delta N_{CDC} < 0$ (\textit{this} track has less hits in the SVD and CDC),
\end{itemize}

Additionally we define an MC truth variable 
\begin{equation}
\label{eq:chi2}
\Delta \chi^2 = \chi^2_{this} - \chi^2_{other},\quad\chi^2 = \sum_{i=x,y,z}\frac{\left(p_i - p_i^{MC}\right)^2}{\sigma(p_i)^2},
\end{equation}
where we compare all components of track momentum to the true values. If $\Delta \chi^2 > 0$, then \textit{this} track has a higher probability of being a duplicate track and should be discarded.

However, it turns out that solving this problem is not as simple as discarding one track and keeping the other one. An additional complication here is that we can have more than one extra track from the same initial particle, which leads to track pairs where both tracks are track duplicates. For example, if we have the following case
\begin{align*}
t_1&: \mathrm{good~track},\\
t_2&: \mathrm{extra~track},\\
t_3&: \mathrm{extra~track},\\
\mathrm{pair}_1&:\left(t_1,t_2\right),\\
\mathrm{pair}_2&:\left(t_1,t_3\right),\\
\mathrm{pair}_3&:\left(t_2,t_3\right),
\end{align*}
where $t_1$ is the original track and $t_2$ and $t_3$ are extra tracks, with $t_3$ being even more duplicate-like with respect to $t_2$. Here tracks $t_2$ and $t_3$ should be discarded while $t_1$ should be kept. We can achieve this if we overwrite existing pair-level information in the tracks for cases where the variable difference $\Delta f$ is more duplicate-like. If we follow the same example, we could fill information about the property $f$ in six different orders. 
\begin{align*}
1.&~\left(t_1,t_2*\right)\quad \to \quad \left(t_1,t_3*\right)\quad \to \quad \left(t_2*,t_3*\right),\\ 
2.&~\left(t_1,t_2*\right)\quad \to \quad \left(t_2*,t_3*\right)\quad \to \quad \left(t_1,t_3*\right),\\ 
3.&~\left(t_1,t_3*\right)\quad \to \quad \left(t_2,t_3*\right)\quad \to \quad \left(t_1,t_2*\right),\\
4.&~\left(t_1,t_3*\right)\quad \to \quad \left(t_1,t_2*\right)\quad \to \quad \left(t_2*,t_3*\right),\\
5.&~\left(t_2,t_3*\right)\quad \to \quad \left(t_1,t_3*\right)\quad \to \quad \left(t_1,t_2*\right),\\
6.&~\left(t_2,t_3*\right)\quad \to \quad \left(t_1,t_2*\right)\quad \to \quad \left(t_1,t_3*\right),
\end{align*}

where the "*" symbol denotes when a track is recognized as a duplicate track with respect to the other track. We see that no matter the order, both $t_2$ and $t_3$ get recognized as duplicate tracks correctly.

\subsubsection{Training the duplicate track MVA}
The training procedure is similar as before. The sample of tracks from duplicate track pairs is now fed into an MVA, which is trained to distinguish duplicate tracks from good tracks. The dataset contains
\begin{itemize}
\item 160146 target candidates,
\item 128568 background candidates,
\end{itemize}
where the definition of target is that the track is a duplicate track. 

The input variables used in this MVA are
\begin{itemize}
\item theta angle of the track momentum,
\item track quantities
	\begin{itemize}
	\item impact parameters $d_0$ and $z_0$,
	\item momentum components $p_T$ and $p_z$
	\item number of hits in the SVD and CDC detectors
	\item track fit $p$-value,
	\end{itemize}
\item pair-level information
	\begin{itemize}
	\item $\Delta d_0$, $\Delta z_0$, $\Delta N_{CDC}$, $\Delta N_{SVD}$, $\Delta p_T$, $\Delta p_z$.  
	\end{itemize}
\end{itemize}

The classifier is shown in Figure \ref{fig:ROE_curl}. The weights from this training are applied to the tracks, where now each track has a certain probability of being a duplicate track. We now compare these values between both tracks in each track pair as
\begin{equation}
\Delta BDT_{final} = BDT_{final}^{this} - BDT_{final}^{other},
\end{equation}

which is again applied to all track pairs and overwritten for tracks which are more duplicate-like.

\begin{figure}[H]
\centering
\captionsetup{width=0.8\linewidth}
\includegraphics[width=\linewidth]{fig/ROECleanup_curl}
\caption{Classifier output of the MVA training for curling track recognition.}
\label{fig:ROE_curl}
\end{figure}

Finally, we select all duplicate tracks which survive the cut 
\begin{equation}
\Delta BDT_{final} > 0
\end{equation}

and discard them from our ROE. We can check the performance of our duplicate track classifier by applying the procedure on a sample of duplicate track pairs and comparing the predicted result with the truth, based on Eq. (\ref{eq:chi2}). Table \ref{tab:rat} shows the performance of the duplicate track recognition in the form of percentages of correctly and incorrectly identified duplicate and original tracks. The model seems to perform well and the event is now considered to be clean of duplicate tracks.

\begin{table}[H]
\centering
\begin{tabular}{|c|c|c|}
\hline
 & Predicted duplicate track & Predicted good track \\
 \hline 
 Duplicate track & $84.63~\%$  & $20.95~\%$  \\
 \hline
 Good track & $15.37~\%$ & $79.05~\%$ \\
 \hline
\end{tabular}
\caption{Ratios of correctly classified and misclassified tracks.}
\label{tab:rat}
\end{table}

\section{Belle clean-up}

The clean-up, used standardly at Belle, is much simpler and relies only on a set of rectangular cuts for neutral particles as well as charged ones. In case of photons, a single cut on photon energy is applied, depending on the region where the photon hit the relevant part of the detector. The photon cuts are summarized in Table \ref{tab:bellegamma}.

\begin{table}[H]
\centering
\begin{tabular}{|c|c|c|c|}
\hline
 & $17^\circ < \theta < 32^\circ$ & $32^\circ < \theta < 130^\circ$ & $130^\circ < \theta < 150^\circ$ \\
 \hline 
 $E_\gamma$ & $> 100\e{MeV}$  & $> 50\e{MeV}$ & $> 150\e{MeV}$  \\
 \hline
\end{tabular}
\caption{Photon selection for the Belle clean-up procedure. Different cuts are applied on photons in different parts of the detector}
\label{tab:bellegamma}
\end{table}

In case of tracks, pairs are selected which satisfy the following criteria:
\begin{itemize}
\item $p_T < 275\e{MeV}/c$,
\item $\Delta p = \vert \textbf{p}_1 - \textbf{p}_2\vert  < 100\e{MeV}/c$,
\item $\cos \theta (\textbf{p}_1,\textbf{p}_2) < 15^\circ$ for same sign,
\item $\cos\theta(\textbf{p}_1,\textbf{p}_2) > 165^\circ$ for opposite sign.
\end{itemize}

Of the two tracks, the one with a larger value of formula in Eq. \ref{eq:belleformula} is discarded. The remaining tracks in the event then need to satisfy the conditions described in Table \ref{tab:belletrack}.
\begin{equation}
\label{eq:belleformula}
\left(\gamma\vert d_0 \vert \right)^2 + \vert z_0 \vert^2, \quad \gamma = 5.
\end{equation}

\begin{table}[H]
\centering
\begin{tabular}{|c|c|c|c|}
\hline
 & $p_T < 250\e{MeV}/c$ & $250\e{MeV}/c < p_T < 500\e{MeV}/c$ & $p_T > 500\e{MeV}/c$ \\
 \hline 
 $\vert d_0 \vert$ & $< 20\e{cm}$  & $< 15\e{cm}$ & $< 10\e{cm}$  \\
 \hline
  $\vert z_0 \vert$ & $< 100\e{cm}$  & $< 50\e{cm}$ & $< 20\e{cm}$  \\
  \hline
 
\end{tabular}
\caption{Photon selection for the Belle clean-up procedure. Different cuts are applied on photons in different parts of the detector}
\label{tab:belletrack}
\end{table}

%TODO: preveri vektorje ce so vedno bold

\section{Clean-up results}

In this section the results of the ROE clean-up are shown. It is clear that cleaning up the event affects the shape of various distribution, especially \vars, which we are most interested in. Since the reconstruction procedure includes cuts on the cleaned-up variables, this also results in an efficiency increase of the reconstructed sample. 

We compare the clean-up setup, defined in this analysis, to the standard clean-up used by Belle, and to a default case, where no clean-up was applied at all. Figure \ref{fig:roeopt} shows distributions of \vars~for various clean-up setups. We see an improvement in resolution as well as an increase in efficiency in both observed variables for the case where the ROE clean-up is performed. Table \ref{tab:roeeff} shows ratios of efficiencies and $FWHM$'s of $\Delta E$ with respect to the default case. While both the Belle and ROE clean-up increase efficiency, only ROE clean-up improves the resolution.

%which is the same as our optimal clean-up, with the exception of the masses of FSP particles in Eq. (\ref{eq:pcharged}), which are, in this case, taken from MC. Mass hypothesis determination of charged FSP particles in Eq. (\ref{eq:pcharged}) based on PID information seems to be quite successful, since the benefits of using true masses from Monte Carlo are very small.

\begin{figure}[H]
\centering
\captionsetup{width=0.8\linewidth}
\includegraphics[width=\linewidth]{fig/roe_opt}
\caption{\vars~distributions for various types of clean-up procedures. For ROE clean-up, the procedure seems to improve resolution as well as increase efficiency, compared to the default case.}
\label{fig:roeopt}
\end{figure}

\begin{table}[H]
\centering
\begin{tabular}{|c|c|c|}
\hline
 & Efficiency ratio & FWHM ratio \\
 \hline 
 Belle clean-up & $121.90~\%$  & $102.60~\%$  \\
 \hline
 ROE clean-up & $123.40~\%$ & $77.92~\%$ \\
 \hline
\end{tabular}
\caption{Comparison of efficiencies and $FWHM$'s of ROE and Belle clean-up setups with respect to the default case (no clean-up).}
\label{tab:roeeff}
\end{table}

Another variable which heavily depends on the clean-up is the charge product of the signal and companion $B$ meson candidate, already defined in Eq. (\ref{eq:chargeprod}), shown in Figure \ref{fig:roe_chargeproduct} for various clean-up procedures. As a cross-check, we can also look at \vars~variables for each value of the charge product. These plots are shown in Figure \ref{fig:roe_split} and they show a clear increase in efficiency and improvement in resolution for the correct value of the charge product in the case of the ROE clean-up. For other values of the charge product there also seems to be an increase in efficiency for both cases of clean-up, but neither help improve the resolution. This supports our choice of signal categorization, defined in Section \ref{sec:event-categorization}, where we select only candidates with the correct value of the charge product.

\begin{figure}[H]
\centering
\captionsetup{width=0.8\linewidth}
\includegraphics[width=\linewidth]{fig/roe_chargeprod}
\caption{Distribution of $B_{sig}$ and $B_{comp}$ charge product for various types of clean-up procedures. Fore ROE clean-up, the procedure seems to increase the number of candidates with the correct value of the product.}
\label{fig:roe_chargeproduct}
\end{figure}

\begin{figure}[H]
\centering
\captionsetup{width=0.8\linewidth}
\includegraphics[width=\linewidth]{fig/roe_split}
\caption{Distributions of $\Delta E$ (top) and $M_{BC}$ (bottom)for various types of clean-up procedures, split by specific values of the charge product. While there seems to be an increase in efficiency for both clean-up procedures in general, ROE-cleanup makes the most improvement to resolution and efficiency for the correct value of the charge product.}
\label{fig:roe_split}
\end{figure}

\section{ROE clean-up validation}

The ROE clean-up seems to perform well on signal MC. It is necessary to make sure that this procedure performs as well on simulated and measured data, which is done in this section.