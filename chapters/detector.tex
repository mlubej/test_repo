\chapter{Experimental Setup}
The data used in this analysis were produced in $e^+e^-$ collisions at the KEKB accelerator and collected with the Belle detector. The experiment was hosted at the High Energy Accelerator Research Organization (KEK) in Tsukuba, Japan. The experiment ran from 1999 to 2010, collecting data at and near the energy of the $\Upsilon(4S)$ resonance. This chapter briefly describes the accelerator and the detector, based on detailed reports from \cite{doi:10.1093/ptep/pts102} and \cite{ABASHIAN2002117}, respectively.


\section{KEKB Accelerator}
KEKB is an asymmetric $e^+e^-$ collider, composed roughly of an electron source and a positron target, a linear accelerator (LINAC), and two separate main rings with a circumference of about $3\e{km}$, as shown in Figure \ref{fig:kekb}. Electrons are first produced by a thermal electron gun and accelerated in the LINAC to an energy of about $8\e{GeV}$. Some of the electrons collide with a tungsten target to produce positrons, which are accelerated in the LINAC to an energy of around $3.5\e{GeV}$. Electron and positron beams are injected into the high- (HER) and low-energy ring (LER), where they collide as bunches of particles at a single interaction point (IP) at a crossing angle of about $22\e{mrad}$. The center-of-mass (CM) energy of the collision corresponds to the mass of the $\Upsilon(4S)$ resonance as
\begin{equation}
E_{CM} = \sqrt{2E_{e^+}E_{e^-}} = m_{\Upsilon(4S)}c^2 \approx 10.58\e{GeV}.
\end{equation}

\begin{figure}[H]
	\centering
	\captionsetup{width=0.8\linewidth}
	\includegraphics[width=0.5\linewidth]{fig/setup/KEKB}
	\caption{Schematic layout of the KEKB accelerator. The HER and the LER are the $e^-$ and the $e^+$ beams, respectively. Four experimental halls (FUJI, NIKKO, OHO and TSUKUBA) are shown.}
	\label{fig:kekb}
\end{figure}

The $\Upsilon(4S)$ state is produced only in a fraction of all collisions, but when it is produced, it predominantly decays to a pair of charged or neutral $B$ mesons. This setup was chosen in accordance with the main goal of the experiment, which was to study $CP$ violation in the $B$ meson system. In other cases, the processes include $e^+e^-$ scattering, also known as Bhabha scattering, two-photon events, muon or tau lepton pair production, and production of $q \bar q$, where $q=u,\,d,\,s$ or $c$. Table \ref{tab:xsec} shows the cross-sections for all mentioned interactions in collisions of $e^+e^-$.
In addition to the nominal CM energy, the experiment collected data also at energies
corresponding to other $\Upsilon(nS)$ resonances, where $n = 1,\,2,\,3,\,5$, and at energies below the resonances.

\begin{table}[H]
	\centering
	\begin{tabular}{c|c}
		Interaction & Cross-section $[\mathrm{nb}]$ \\ 
		\toprule
		$\Upsilon(4S) \to B \bar B$ & $1.2$ \\
		$q \bar q,~q \in [u,d,s,c]$ & $2.8$ \\
		$\mu^+\mu^-,~\tau^+\tau^-$ & $1.6$ \\
		Bhabha scattering (within detector acceptance)& $44$ \\
		Other QED processes (within detector acceptance)& $\sim 17$ \\
		\midrule
		Total & $\sim 67$ \\
		\bottomrule
	\end{tabular}
	\caption{Cross-sections for various physics processes at $\Upsilon(4S)$ resonance energy~\cite{ABASHIAN2002117}.}
	\label{tab:xsec}
\end{table}

KEKB achieved the world-record in the instantaneous luminosity of $2.11\E{34}\e{cm^{-2}s^{-1}}$, twice as large as the design value. The total dataset, collected with the Belle detector, amounts to the total integrated luminosity of $1041\e{fb^{-1}}$. Out of the full Belle dataset about $711\e{fb^{-1}}$ of data were taken at the $\Upsilon(4S)$ energy of $10.58\e{GeV}$, corresponds to about $771\E{6}$ $B \bar B$ meson pairs.

\section{Belle Detector}
The Belle detector is a magnetic mass spectrometer which covers a large solid angle. It is designed to detect products of $e^+e^-$ collisions. The detector is configured around a $1.5\e{T}$ superconducting solenoid and iron structure surrounding the interaction point (IP). The 4-momentum of the decaying $B$ mesons and its daughter particles are determined via a series of sub-detector systems, which are installed in an onion-like shape, as shown in Figure \ref{fig:bdet}. Short-lived particle decay vertices are measured by the silicon vertex detector (SVD), situated outside of a cylindrical beryllium beam pipe. Long-lived charged particle momentum is measured via tracking, which is performed by a wire drift chamber (CDC). Particle identification is provided by energy-loss measurements in CDC, aerogel Cherenkov counters (ACC), and time-of-flight counters (TOF), situated radially outside of CDC. Particles producing electromagnetic showers deposit energy in an array of CsI(Tl) crystals, known as the electromagnetic calorimeter (ECL), which is located inside the solenoid coil. Muons and $K_L$ mesons (KLM) are identified by arrays of resistive plate counters in the iron yoke on the outside of the coil. 

\begin{figure}[H]
	\centering
	\captionsetup{width=0.8\linewidth}
	\includegraphics[width=0.8\linewidth]{fig/setup/Belle_detector}
	\caption{Schematic layout of the Belle detector with the corresponding sub-detectors \cite{ABASHIAN2002117}.}
	\label{fig:bdet}
\end{figure}

The coordinate system of the Belle detector originates at the IP, with the $z$ axis pointing in the opposite direction of the positron beam, the $x$ axis pointing horizontally out of the ring, and the $y$ axis being perpendicular to the aforementioned axes. The electron beam crosses the positron beam at an angle of about $22^\circ$. The polar angle $\theta$ covers the region between $17^\circ \leq \theta \leq 150^\circ$, while the cylindrical angle $\varphi$ covers the full $360^\circ$ range, amounting to a solid angle coverage of about $92\%$.


%In our analysis, key features of the Belle detector are as follows.
%•
%Good particle detection efficiency for 4-
%π
%region to collect all the particles in an
%e
%+
%e
%−
%collision.  This is important not only for the good signal reconstruction efficiency,
%but also for background reduction since un-detected particles from various
%B
%decay
%modes are the main source of the background.
%•
%Good reconstruction efficiency for low-momentum particles.  Since decay particles
%from
%D
%∗
%mesons and
%τ
%leptons are produced in a chain of multi-body decays, their
%momenta are about 0.7 GeV
%/c
%on average and lower than 1.5 GeV
%/c
%.
%•
%Good  momentum  resolution  in  the  tracking  devices  and  energy  resolution  in  the
%ECL.
%•
%Good PID performance for charged particles.

%\subsection{Extreme forward calorimeter (EFC)}

\subsection{Silicon Vertex Detector}
SVD is the inner-most part of the Belle detector and its purpose is the determination of the decay vertices of decaying particles. The precision of the subsystem is about $100\e{\mu m}$, which is important for measuring the difference in $z$-vertex positions of the $B$ mesons in time-dependent $CP$ violation studies. The main part of the SVD are the double-sided silicon detectors (DSSD). With their thin profile and parallel silicon strips on both sides, they provide two-dimensional hit information of the charged particle and are perfect for a small-scale device which acts with high precision.

During the data taking period, two configurations of the SVD have been used. The first, SVD1, has three layers of DSSD detectors, positioned at $30$, $45.5$ and $60\e{mm}$ away from the IP. They compose a ladder-like structure, covering the polar angle of $23^\circ < \theta < 140^\circ$. This configuration was used from the beginning of the experiment until 2003, when a dataset of about $1.52\E{8}$ pairs of $B \bar B$ mesons was recorded. Due to problems with radiation hardness, a new configuration was used, SVD2, which was operational until the end of data taking, measuring about $6.20\E{8}$ pairs of $B \bar B$ mesons. The SVD2 has 4 layers of DSSD detectors positioned at $20$, $43.5$, $70$ and $80\e{mm}$ away from the IP and covers the polar angle of $17^\circ < \theta < 150^\circ$. The first layer was moved closer to the IP, which greatly improved the sub-system precision, since the multiple-Coulomb scattering affects the resolution more as the distance from the IP increases. The front and the side views of the SVD2 are shown in Figure \ref{fig:SVD_layout}.

\begin{figure}[H]
	\centering
	\captionsetup{width=0.8\linewidth}
	\includegraphics[width=\linewidth]{fig/setup/SVD_layout}
	\caption{Front (left) and side (right) view of the SVD detector with the SVD2 configuration. The front view also shows the inner wires of the Central Drift Chamber \cite{haba2004letter}.}
	\label{fig:SVD_layout}
\end{figure}

The efficiency of the SVD was determined as a fraction of CDC tracks within the SVD acceptance with associated SVD hits, needed for the $B$ meson reconstruction. The average efficiency is found to be around $98\%$ and is in agreement with simulation. SVD performance is also determined based on the resolutions of the impact parameter $z$ and $r\phi$, obtained from cosmic ray data. The momentum and angular dependence of the impact parameters resolution is well represented as a function of the pseudo-momentum for the SVD2 by the following parametrization
\begin{align}
\sigma_z = 28\e{\mu m}\oplus \frac{32\e{\mu m}}{\left(p/(1\e{GeV}/c)\right)}\frac{1}{\beta \sin^{5/2}\theta},\\
\sigma_{r\phi} = 22\e{\mu m}\oplus \frac{36\e{\mu m}}{\left(p/(1\e{GeV}/c)\right)}\frac{1}{\beta \sin^{3/2}\theta},
\end{align}
where pseudo-momentum is defined as $p\beta \sin^{5/2}\theta$ for the $z$ direction and as $p\beta \sin^{3/2}\theta$ for the $r\phi$ plane, $p$ is the particle momentum, $\theta$ is the polar angle, and $\beta=v/c$. An advantage of the smaller distance between the IP and the first DSSD layer in SVD2 is clearly seen in Figure \ref{fig:SVD_performance}.

\begin{figure}[H]
	\centering
	\captionsetup{width=0.8\linewidth}
	\includegraphics[width=\linewidth]{fig/setup/SVD_performance_1}
	\caption{Impact parameter resolutions of $z$ (left) and $r\phi$ (right) coordinates for the SVD1 and SVD2 configuration of the vertex detector \cite{haba2004letter}.}
	\label{fig:SVD_performance}
\end{figure}


\subsection{Central Drift Chamber}

CDC is a large-volume tracking device located in the central part of the Belle detector. It is crucial for measurements of the particle trajectories and momenta, but also serves as a particle identification device (PID). It has a cylindrical structure with a radius of $88\e{cm}$, length of $2.4\e{m}$ and acceptance equal to the one of SVD2. The chamber has a total of $8400$ wires, which are positioned in $50$ layers and describe a nearly square wire configuration. There are two types of wires -- field wires for producing the electrical field, and sense wires for detecting the particles. Odd-numbered wire layers are oriented in the $z$ direction and provide a measurement of the transverse momentum $p_t$, while even-numbered wires are inclined with respect to the $z$ axis by a small angle of $\pm 50\e{mrad}$ to allow for the measuring of the polar angle of the track. The wire configuration is shown in Figure \ref{fig:CDC_layout}. The space between the wires is filled with a gas mixture of $1:1$ helium-ethane, a low-$Z$ gas, in order to minimize multiple-Coulomb scattering contributions to the momentum resolution, since the majority of particles in $B$ events have a momentum lower than $1\e{GeV}/c$. It also has a small cross section for the photoelectric effect, which is important for reducing background electrons induced by the synchrotron radiation from the beam. 

\begin{figure}[H]
	\centering
	\captionsetup{width=0.8\linewidth}
	\includegraphics[width=0.6\linewidth]{fig/setup/CDC_layout}
	\caption{Cell structure of CDC \cite{ABASHIAN2002117}.}
	\label{fig:CDC_layout}
\end{figure}

Charged particles, which pass the CDC wire frame, cause gas ionization. The produced electrons drift toward the sense wires with high acceleration, due to the strong electric field close to the wire. The accelerated electrons collide with the gas molecules and produce secondary, tertiary, and so on, ionizations, which results in an electron avalanche -- a process, which increases the signal by many orders of magnitude. The primary electrons also have a specific drift velocity, which enables us to relate the measured pulse height and drift time to the energy deposit of the particle, as well as the distance from the sense wire. This information is important for calculating the energy loss $\mathrm{d}E/\mathrm{d}x$. $\mathrm{d}E/\mathrm{d}x$, as a function of momentum, differs for different particles, as shown in Figure \ref{fig:dEdx}. This allows for identification purposes, specifically for kaons and pions. In the momentum region of less than $0.8\e{GeV}/c$, $\mathrm{d}E/\mathrm{d}x$ enables a separation between kaons and pions with at least $3\sigma$. The resolution of the transverse momentum measurement with the CDC is a function of the transverse momentum itself, as well as the particle velocity, and is parametrized as
\begin{equation}
\sigma(p_T)/p_T = \frac{0.201\%~p_T}{1\e{GeV}/c} \oplus \frac{0.290\%}{\beta}.
\end{equation}

\begin{figure}[H]
	\centering
	\captionsetup{width=0.8\linewidth}
	\includegraphics[width=0.6\linewidth]{fig/setup/dEdx}
	\caption{Measured $\mathrm{d}E/\mathrm{d}x$ as a function of particle momentum. The red lines show the expected distribution for different types of particles \cite{ABASHIAN2002117}.}
	\label{fig:dEdx}
\end{figure}

\subsection{Time-of-Flight Counter}

The purpose of the TOF subdetector is particle identification in the momentum region of $0.8\e{GeV}/c < p < 1.2\e{GeV}/c$, especially for kaons and pions. There are 64 TOF modules in the barrel region, covering the polar angle of $33^\circ < \theta < 121^\circ$. One TOF module consists of two long polyvinyl toluene-based plastic scintillator bars, 4 fine-mesh photo-multiplier tubes (PMT) at the 4 ends of the bars, and a trigger scintillation counter, where the latter provides additional trigger information. TOF measures the time interval between the $e^+e^-$ collision and the passage of the particle through it. The mass of a particle can be inferred via the relation
\begin{equation}
m^2 = \left( \frac{1}{\beta^2}-1\right)p^2 = \left( \frac{T^2c^2}{L^2}-1\right)p^2,
\end{equation}
where $T$ is the measured time interval, $L$ is the charged particle trajectory length from the IP to TOF, and $p$ is the charged particle momentum, determined by SVD and CDC. The resulting mass distribution for charged tracks identified by TOF in hadron events is shown in Figure \ref{fig:TOF_mass}, where clear peaks corresponding to pions, kaons, and protons can be seen. To achieve the good discrimination between kaons and pions, a time-of-flight resolution of less than $100\e{ps}$ is needed for particles with momentum below $1.2\e{GeV}/c$, which encompasses $90\%$ of the particles produced in $\Upsilon(4S)$ decays. The identification power can also be determined in the form of $\pi^\pm/K^\pm$ separation significance as a function of the particle momentum, shown in Figure \ref{fig:TOF_separation}. A clear separation of about $2\sigma$ is achieved for particle momenta up to $1.25\e{GeV}/c$.

\begin{figure}[H]
	\centering
	\captionsetup{width=0.8\linewidth}
	\includegraphics[width=0.6\linewidth]{fig/setup/TOF_mass}
	\caption{Mass distribution from TOF measurements for particle momenta below $1.2\e{GeV}/c$ \cite{ABASHIAN2002117}.}
	\label{fig:TOF_mass}
\end{figure}

\begin{figure}[H]
	\centering
	\captionsetup{width=0.8\linewidth}
	\includegraphics[width=0.6\linewidth]{fig/setup/TOF_separation}
	\caption{$\pi^\pm/K^\pm$ separation by TOF \cite{ABASHIAN2002117}.}
	\label{fig:TOF_separation}
\end{figure}


\subsection{Aerogel Cherenkov Counter}
TOF is not capable of performing efficient PID above $1.2\e{GeV}/c$ momentum since $\beta$ is almost equal to 1. For higher momenta in the region of $1.0\e{GeV}/c < 4.0\e{GeV}/c$ the ACC is introduced. It is a threshold-type Cherenkov counter, which utilizes the fact that a particle emits Cherenkov light if the speed of the particle exceeds the speed of light in the passing medium. ACC is introduced in the barrel region with 960 separate modules, covering a polar angle of $34^\circ < \theta < 127^\circ$, and with 228 modules in the forward end-cap region, with the polar angle coverage of $17^\circ < \theta < 34^\circ$. Each ACC module consists of an aluminum encased block of silica aerogel and one or two fine-mesh PMTs encased on each block to detect Cherenkov light pulses. Due to the polar angle dependence of the particles momenta, 6 different refractive indices are chosen for the aerogel material, ranging from $1.010$ up to $1.030$, and are controlled within $3\%$ precision. The layout of the ACC is shown in Figure \ref{fig:ACC_layout}.
\begin{figure}[H]
	\centering
	\captionsetup{width=0.8\linewidth}
	\includegraphics[width=\linewidth]{fig/setup/ACC_layout}
	\caption{Cross-sectional view of the Belle detector with some of the highlighted detector components. The barrel ACC is located between the CDC and TOF, while the endcap ACC is located in the forward endcap, behind the ECL crystals \cite{ABASHIAN2002117}.}
	\label{fig:ACC_layout}
\end{figure}
The threshold velocity $\beta $to irradiate Cherenkov light, for a given particle of mass $m$, is given by
\begin{equation}
\beta \geq \frac{1}{n},
\end{equation}
where $n$ is the refractive index of the medium. The refractive indices in the ACC are chosen such that due to a difference in masses, for a given interval of momenta, pions emit Cherenkov light, while kaons do not. Using the PID information from ACC, as well as other sub-systems, the electron identification efficiency, in the momentum range above $1\e{GeV}/c$, is larger than $90\%$, while the pion fake rate -- the probability of wrongly identifying pions as electrons -- is around $0.2$ - $0.3\%$. Similarly for kaons, kaon PID efficiency is equal to $80\%$ for most of the momentum region up to $4\e{GeV}/c$, while pion fake rate remains below $10\%$. Figure \ref{fig:ACC_eff} shows the electron and kaon efficiencies, and the corresponding pion fake rates as a function of the particle's momentum.

\begin{figure}[H]
	\centering
	\captionsetup{width=0.8\linewidth}
	\includegraphics[width=0.48\linewidth]{fig/setup/ACC_eff1}
	\includegraphics[width=0.48\linewidth,trim = 0cm -4cm 0cm 0cm]{fig/setup/ACC_eff2}
	\caption{Electron identification efficiency and fake rate for charged pions (left) and similarly for kaons (right). Note the different scales for the electron efficiency and fake rate in the left plot \cite{ABASHIAN2002117}.}
	\label{fig:ACC_eff}
\end{figure}


\subsection{Electromagnetic Calorimeter}
Measurement of the position and energy deposit of particles is performed in the ECL, especially for electrons and photons. The latter are electrically neutral and, therefore, not detected in any of the previously mentioned subdetectors. The ECL also provides complimentary particle identifications for the separation of electrons and pions. The calorimeter consists of a highly segmented array of thallium-doped cesium iodide (CsI(Tl)) in the form of tower-shaped crystals, each pointing towards the IP. Each crystal is about $30\e{cm}$ long with a width from $44.5\e{mm}$ to $65\e{mm}$ in the barrel, and from $44.5\e{mm}$ to $82\e{mm}$ in the end-caps. Out of a total of 8736 crystals with a total mass of about $43\e{tons}$, 6624 of them are positioned in the barrel region and 1152 (960) in the forward (backward) end-caps. The inner radius of the barrel section is about $1.25\e{m}$, while the end-caps are positioned at $-1.0\e{m}$ and $2.0\e{m}$ from the IP in direction of the $z$ axis. The polar angle coverage of the barrel region is $32.2^\circ < \theta < 128.7^\circ$, and for the forward and backward end-caps $12.4^\circ < \theta < 31.4^\circ$ and $130.7^\circ < \theta < 155.1^\circ$, respectively. Figure \ref{fig:ECL_layout} shows the layout of the barrel and end-cap ECL. 
\begin{figure}[H]
	\centering
	\captionsetup{width=0.8\linewidth}
	\includegraphics[width=\linewidth]{fig/setup/ECL_layout}
	\caption{Overall configuration of the ECL \cite{ABASHIAN2002117}.}
	\label{fig:ECL_layout}
\end{figure}
When an electron or a photon hits a crystal, it produces an electromagnetic shower, a result of the bremsstrahlung and pair-production effects. Heavier charged particles do not interact in the same way and deposit only a small amount of energy by ionization effects. The information from the ECL, compared with momentum measurements provided by the CDC, enables the identification of electrons. The distribution of the deposited energy for different particles is shown in Figure \ref{fig:ECL_deposit}. The probability of misidentifying an electron as a pion is approximately $5\%$ for momenta less than $1\e{GeV}/c$ and less than $1\%$ for momenta above $2\e{GeV}/c$.

\begin{figure}[H]
	\centering
	\captionsetup{width=0.8\linewidth}
	\includegraphics[width=0.6\linewidth]{fig/setup/ECL_deposit}
	\caption{Distribution of the energy deposit by electrons and charged pions at $1\e{GeV}/c$ momentum \cite{ABASHIAN2002117}.}
	\label{fig:ECL_deposit}
\end{figure}

For ECL calibration, $e^+e^- \to e^+e^-$ and $e^+e^- \to \gamma\gamma$ events were used. The average energy resolution was achieved to be $1.7\%$ for the barrel ECL, and $1.74\%$ and $2.85\%$ for the forward and backward ECL, respectively, as shown in Figure \ref{fig:ECL_resolution}. These value are in good agreement with Monte Carlo predictions. Worse energy resolution in the backward end-cap is due to the lower photon energy, which results in larger scattering in the material in front of the calorimeter \cite{haba2004letter}. The energy resolution as a function of the energy can be described via the following relation
\begin{equation}
\frac{\sigma_E}{E} = \frac{0.0066\%}{\left(E/1\e{GeV}\right)}\oplus\frac{1.53\%}{\left( E/1\e{GeV}\right)^{1/4}}\oplus 1.18\%,
\end{equation}
while the resolution of the position measurement is
\begin{equation}
\sigma_{\mathrm{pos}} = 0.27\e{mm}+\frac{3.4\e{mm}}{\left( E/1\e{GeV}\right)^{1/2}} + \frac{1.8\e{mm}}{\left( E/1\e{GeV}\right)^{1/4}}.
\end{equation}

\begin{figure}[H]
	\centering
	\captionsetup{width=0.8\linewidth}
	\includegraphics[width=0.8\linewidth]{fig/setup/ECL_resolution}
	\caption{Reconstructed energy distribution for $e^+e^- \to \gamma\gamma$ events for overall, barrel, forward, and backward end-cap ECL \cite{ABASHIAN2002117}.}
	\label{fig:ECL_resolution}
\end{figure}

\subsection{\texorpdfstring{$K_L^0/\mu$}{KL0/µ} Detector}
The KLM detector is used for detection of high-penetration particles such as $K_L^0$ and $\mu$ with momenta larger than $0.6\e{GeV}/c$. The setup covers the polar angle of $20^\circ < \theta < 155^\circ$. Detection of $K_L^0$ particles is troublesome, since they are neutral and have a small material interaction probability, therefore a lot of material is needed in the KLM. To provide detection of both kinds of particles, hadronic and neutral, as well as electromagnetically and hadronically interacting, the KLM is constructed as a sampling calorimeter, which consists of 15 layers of $3.7\e{cm}$ thick resistive-plate counters (RPC) with 14 layers of $4.7\e{cm}$ thick iron plates between them. A single RPC module consists of two parallel plate electrodes, two glass panels, and gas in between. A charged particle passing the gas gap initiates a local discharge of the plates, which in turn induces signal to record the time and location of ionization. This is possible since the resistivity of the glass surface is high, so the discharge occurs locally. Hadrons interacting with the iron plates may produce a shower of ionizing particles, which are then also detected by the RPCs. The KLM is located outside of the superconducting solenoid. The iron plates of the KLM serve a dual role; as the necessary material for hadronic interactions, as well as the flux return for the magnetic field. Figure \ref{fig:KLM_layer} shows a cross-section of an RPC superlayer, consisting of an RPC pair.

\begin{figure}[H]
	\centering
	\captionsetup{width=0.8\linewidth}
	\includegraphics[width=0.5\linewidth]{fig/setup/KLM_layer}
	\caption{Cross-section of an RPC superlayer, consisting of an RPC pair \cite{ABASHIAN2002117}.}
	\label{fig:KLM_layer}
\end{figure}

The $K_L^0$ particle can be distinguished from other charged hadrons by the fact that they have no matched track in the CDC. The flight direction can be inferred from the hit locations in the consecutive RPCs. Tracks of charged particles measured in CDC are extrapolated into KLM, and clusters within $15^\circ$ of an extrapolated charged particle track are excluded from $K_L^0$ cluster candidates. On the other hand, muons with matched CDC tracks are able to reach the KLM if their momentum is larger than $0.5\e{GeV}/c$. They do not interact strongly and do not produce hadronic showers in the KLM. Figure \ref{fig:KLM_eff} (left) shows the number of neutral clusters per event and a Monte Carlo simulation of the predicted number of $K_L^0$ clusters per event. The average number of $K_L^0$ clusters per event is $0.5$. The agreement with the prediction gives us the confidence that the detector and our reconstruction software are performing correctly. Figure \ref{fig:KLM_eff} (right) shows the muon detection efficiency as a function of the momentum, shown for a likelihood cut of $0.66$, where muon likelihood is based on the comparison of the measured range of a particle with the predicted range for a muon. Using $K_S \to \pi^+\pi^-$ events, a muon identification efficiency of better than $90\%$ is determined, with a pion fake rate of less than $5\%$ for particles with momenta larger than $1.5\e{GeV}/c$ and an applied likelihood cut of $0.66$.

\begin{figure}[H]
	\centering
	\captionsetup{width=0.8\linewidth}
	\includegraphics[width=0.48\linewidth,trim = 0cm -1.5cm 0cm 0cm]{fig/setup/KLM_clusters}
	\includegraphics[width=0.48\linewidth]{fig/setup/KLM_efficiency}
	\caption{Number of neutral clusters per event in KLM (left) and muon detection efficiency as a function of momentum in KLM (right) \cite{ABASHIAN2002117}.}
	\label{fig:KLM_eff}
\end{figure}

Cosmic ray events have been used to determine the efficiency and resolution of the KLM, with an overall efficiency typically above $98\%$. The temporal and spatial resolutions of the KLM are few$\e{ns}$ and about $1.2\e{cm}$, respectively. The latter corresponds to an angular resolution from the interaction point of better than $10\e{mrad}$.

In order to do detector calibration and proper luminosity measurements, we need to accumulate samples of Bhabha and $\gamma\gamma$ scattering. Otherwise, as shown in Table \ref{tab:xsec}, the cross-section for physics events of interest is reasonably small. During normal operation the total event rate is around $200\e{Hz}$, which is well below the data acquisition (DAQ) limit of $500\e{Hz}$. Out of this rate, $100\e{Hz}$ are physically interesting events, which include also two-photon events, Bhabha scattering, and $\mu$ pair production, besides hadronic events from $B \bar B$ pair events. In order to discard events which are not interesting for physics analyses, we use a trigger system by appropriately applying restrictive conditions. The next section describes the necessary procedures and equipment to successfully do so.

\subsection{Trigger System}
The trigger system operates by immediately eliminating events that are not of interest, so that the amount of stored data is within the $500\e{Hz}$ frequency limit, while the efficiency for physics events of interest is kept high. Events which pass the triggers are then stored, otherwise discarded. The Belle trigger system consists of three stages, Level-1 (L1) online hardware trigger, Level-3 (L3) online software trigger and Level-4 (L4) offline software trigger.

L1 trigger is the first stage of the trigger system, which consists of multiple sub-detector triggers, all connected to a central trigger system called the Global Decision Logic (GDL), as schematically shown in Figure \ref{fig:TRG_GDL}. Each sub-detector trigger works on a principle of either a track trigger or an energy trigger. In the former case, the triggers discard events not meeting conditions based on the number of reconstructed tracks or track hits, while the latter is based on the total energy deposit and counting of crystal hits. Each sub-detector processes the event information and provides it to the GDL, where all the information is combined and the current event is characterized. The information from the sub-detector triggers reaches the GDL within $1.85\e{\mu s}$ after the collision, and the final trigger signal is provided within a fixed $2.2\e{\mu s}$ latency. The combined efficiency from the L1 trigger is greater than $99.5\%$ for hadronic events.

\begin{figure}[H]
	\centering
	\captionsetup{width=0.8\linewidth}
	\includegraphics[width=0.8\linewidth]{fig/setup/TRG_GDL}
	\caption{The Level-1 trigger system for the Belle detector \cite{ABASHIAN2002117}.}
	\label{fig:TRG_GDL}
\end{figure}

After passing L1 trigger, the L3 discards background events from the software-wise perspective. L3 is an online software trigger which performs a simple, but fast reconstruction of the event. Events with at least one track satisfying the impact parameter condition $\vert\mathrm{d}z \vert < 5.0\e{cm}$ and with a total energy deposit in the ECL more than $1\e{GeV}/c$ are selected. The L3 trigger reduces the event rate by $50\%$, with a $99\%$ efficiency for hadronic events.

After passing the L3 trigger, the events are recorded on tapes. However, these data still contain many events from the beam background. To reduce the background events even further, they are required to pass the L4 offline software filtering. At the same time, a high efficiency for signal events is still required. Events must satisfy the following conditions
\begin{itemize}
	\item have at least one track with $p_T > 300\e{MeV}/c$, impact parameters $\vert \mathrm{d}r \vert < 1.0\e{cm}$, and $\vert \mathrm{d}z \vert < 4.0\e{cm}$,
	\item have a total energy deposit in the ECL greater than $4\e{GeV}$.
\end{itemize}
Approximately $27\%$ of triggered events are passed through L4 while keeping an almost $100\%$ efficiency for hadronic events. Events that pass the L4 trigger are fully reconstructed and stored to the file system. Overall, the efficiency of hadronic events after all trigger stages is measured to be more than $99\%$.



